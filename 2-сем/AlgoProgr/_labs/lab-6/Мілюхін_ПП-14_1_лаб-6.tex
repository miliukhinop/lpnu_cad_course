%-shell-escape, якщо використовуєте minted
\documentclass[a4paper, 12pt, oneside]{extarticle}
\input{$HOME/Templates/lpnu_doc_templates/settings/preamble.tex}
% якщо домахуються за Times New Roman, то
% використовуєте xelatex і цей файл:
\input{$HOME/Templates/lpnu_doc_templates/settings/font_styles.tex}
\input{$HOME/Templates/lpnu_doc_templates/settings/minted_settings.tex}

\newcommand\Variant{4}
\newcommand\Date{28.04.\the\year}
\newcommand\Discipline{Алгоритмізація та програмування, частина 2}
\newcommand\Instructor{Кулешник Я. Ф.}

\newcommand\Type{\Lab}
\newcommand\Number{6}
\newcommand\Topic{Створення i ведення збалансованих бінарних дерев}

\begin{document}
\Margins

\input{$HOME/Templates/lpnu_doc_templates/parts/header.tex}

Ознайомитися iз способом подання даних в оперативнiй пам'ятi ЕОМ у
виглядi збалансованих бінарних (АВЛ) дерев. Оволодiти методами роботи iз
збалансованими бінарними деревами.

\section*{Індивідуальне завдання}

Написати програму, яка створює збалансоване бінарне дерево. Написати
процедуру, яка видаляє з дерева всі парні елементи.

\section*{Етапи розв'язку}

Проаналізував завдання, реалізував AVL-дерево.
Написав функцію для видалення парних елементів.

\subsection*{Код програми}

\inputminted{c++}{avl.cpp}

\subsection*{Результат виконання програми}

\verbatiminput{output}

\section*{Висновок}

Виконуючи цю лабораторну роботу, я навчився створенню AVL-дерев
мовою C++.

\section*{Відповіді на контрольні запитання}
\begin{itemize}
	\question Які дерева називаються виродженими?
	\answer незбалансовані

	\question Яке бінарне дерево називається ідеально збалансованим?
	\answer Таке, в якому кількість вершин у лівому та правому піддеревах відрізняється максимум на 1.

	\question Як можна оцiнити кiлькiсть варiантiв структур бiнарних дерев? Скільки серед них будуть ідеально збалансованими?
		\answer Кiлькiсть варiантiв структур бiнарних дерев можна приблизно оцiнити за допомогою формули Стирлiнґа ($4^n/n*3/2$, n --- кількість ключів), ідеально збалансованих із них дуже мало.

	\question Чому на практиці ідеально збалансовані дерева пошуку використовуються рідко?
	\answer Тому що це дуже затратно та не дає достатньо великого виграшу в ефективності пошуку.

	\question Що так АВЛ-дерево?
	\answer Дерево, у якому реалізоване балансування за різницею
	\question Як визначити показник зблансованостi вузла?
	\answer Відняти висоту його правого піддерева від висоти лівого або навпаки.
	\question Як здiйснюється додавання вузла до АВЛ-дерева? Як виконуються процедури збалансування?
	\answer Додання вузла здійснюється так само, як і у звичайного дерева пошуку, але після нього
		відбувається перевірка різниці висот піддерев, і якщо вона дорівнює 2, виконується балансування.
		Балансування реалізують за допомогою поворотів навколо вузлів дерева.
	\question Як здiйснюється видлення вузла з АВЛ-дерева?
		\answer Якщо вузол --- листок, або одного з його піддерев не існує, то достатньо просто повернути вказівник на друге піддерево. Якщо ж у вузла є два піддерева, то можна, наприклад, знайти мінімальний елемент правого піддерева й замінити вузол, що видаляється, на нього. Далі потрібно виконати балансування.
	\question Як здійснити доступ до елементу за індексом у АВЛ-дереві?
	\answer Так само, як у бінарному дереві пошуку.
\end{itemize}

\end{document}
