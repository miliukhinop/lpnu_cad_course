\documentclass[a4paper, 12pt, oneside]{extarticle}
%xelatex
%-shell-escape % якщо використовуєте minted
\input{$HOME/Templates/lpnu_doc_templates/settings/preamble.tex}
\input{$HOME/Templates/lpnu_doc_templates/settings/minted_settings.tex}
% якщо домахуються дуже за Times New Roman, то
% використовуєте xelatex і цей файл:

\newcommand\Variant{4}
\newcommand\Date{10.04.\the\year}
\newcommand\Discipline{Алгоритмізація та програмування, частина 2}
\newcommand\Instructor{Кулешник Я. Ф.}

\newcommand\Type{\Lab}
\newcommand\Number{4}
\newcommand\Topic{TOPIC}

\begin{document}
\Margins

\input{$HOME/Templates/lpnu_doc_templates/parts/header.tex}

Мета роботи – ознайомитись із особливостями застосування динамічних
об'єктів складної структури:  списками, стеками та чергами; з операціями, які
виконуються над елементами цих об'єктів. Набути практичних навичок
програмування з використанням динамічних об'єктів складної структури.

\section*{Індивідуальне завдання}

4. Сформувати однонаправлений список і видалити всі елементи, які
знаходяться між елементами зі значеннями “X” та “Z”. Вивести на друк
обидва списки.

\section*{Етапи розв'язку}

\subsection*{Код програми}

\subsubsection*{Файл list.h}
\inputminted{c++}{program/list.h}

\subsubsection*{Файл main.cpp}
\inputminted{c++}{program/main.cpp}

\subsection*{Результат виконання програми}

\begin{verbatim}
[sasha@honeypot ~uni/AlgoProgr/lab4/program]$ ./main
0       X
1
2
3
4
5
6       Z
7
8
9
10
11
Видалення символів між X та Z
`````````````````````````````
Знайдено X за індексом 0
Знайдено Z за індексом 6
0       X
1       Z
2
3
4
5
6
\end{verbatim}

\section*{Висновок}

У цій роботі я ознайомився з реалізацією однонаправлених
списків мовою C++.

\section*{Відповіді на контрольні запитання}
\begin{itemize}
	\question Яка особливість динамічного рядка?
	\answer Особливість динамічного рядка полягає в тому, що його довжина може динамічно змінюватись під час виконання програми. Це означає, що рядок може збільшуватись або зменшуватись в залежності від потреб програми.

	\question Що таке список і навіщо він використовується?
	\answer Список - це структура даних, що зберігає послідовність елементів. Використовується для зберігання великої кількості елементів, які можна легко додавати, видаляти та редагувати.

	\question  Чим відрізняється однонаправлений список від двонаправленого списку?
	\answer Однонаправлений список дозволяє переходити від початку до кінця, але не дозволяє переходити назад. Двонаправлений список має можливість переходу як вперед, так і назад.

	\question У чому особливість кільцевого списку?
	\answer Особливість кільцевого списку полягає в тому, що останній елемент списку посилається на перший елемент, утворюючи тим самим замкнутий кільцевий цикл.

	\question  Як видалити елемент у двонаправленому списку?
	\answer Щоб видалити елемент у двонаправленому списку, потрібно спочатку знайти його. Потім можна видалити елемент, змінивши посилання на попередній та наступний елементи так, щоб вони посилалися один на одного, обхідний елемент видаляється з пам'яті.

	\question  Чим відрізняється стек від черги?
	\answer Стек та черга - це дві різні структури даних, які використовуються для зберігання та управління об'єктами.
Основна відмінність між стеком та чергою полягає в тому, як вони керуються додаванням та видаленням елементів.
Стек працює за принципом "останній ввійшов - перший вийшов" (LIFO - last in, first out), що означає, що останній елемент, який доданий до стеку, буде першим, що буде вилучено. Стек можна уявити як стопку книг, де останній доданий том лежить сверху, і його можна взяти першим.
Черга ж працює за принципом "перший ввійшов - перший вийшов" (FIFO - first in, first out), що означає, що перший елемент, який доданий до черги, буде першим, що буде вилучено. Чергу можна уявити як чергу людей, де перший ввійшов до черги, буде першим, хто отримає послугу.
Отже, відмінності між стеком та чергою полягають у порядку видалення елементів: в стеку вилучається останній доданий елемент, а в черзі - перший доданий елемент.
\end{itemize}

\end{document}
