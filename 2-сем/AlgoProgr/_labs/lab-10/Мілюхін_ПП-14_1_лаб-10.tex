%-shell-escape, якщо використовуєте minted
\documentclass[a4paper, 12pt, oneside]{extarticle}
\input{$HOME/Templates/lpnu_doc_templates/settings/preamble.tex}
% якщо домахуються за Times New Roman, то
% використовуєте xelatex і цей файл:
\input{$HOME/Templates/lpnu_doc_templates/settings/font_styles.tex}
\input{$HOME/Templates/lpnu_doc_templates/settings/minted_settings.tex}

\newcommand\Variant{4}
\newcommand\Date{1.05.\the\year}
\newcommand\Discipline{Алгоритмізація та програмування, частина 2}
\newcommand\Instructor{Кулешник Я. Ф.}

\newcommand\Type{\Lab}
\newcommand\Number{10}
\newcommand\Topic{Алгоритми сортування}

\begin{document}
\Margins

\input{$HOME/Templates/lpnu_doc_templates/parts/header.tex}
Вивчення та практична реалізація алгоритмів сортування.

\section*{Індивідуальне завдання}

\begin{tabular}{c|c c c c}
	4 & 4 & 8 & 12 & 16 \\
	\hline
	Сортування & Шелла & розрядами & підрахунком & бінарними вставками
\end{tabular}

\section*{Етапи розв'язку}

Проаналізував та реалізував алгоритми, вказані
в індивідуальному завданні.

\section*{Програмні коди}

\subsection*{Сортування Шелла}

\inputminted{c++}{shell_sort.cpp}
\subsection*{Результат виконання програми}
\inputminted{c++}{shell_sort_out}

\subsection*{Сортування розрядами}

\inputminted{c++}{radix_sort.cpp}
\subsection*{Результат виконання програми}
\inputminted{c++}{radix_out}

\subsection*{Сортування підрахунком}

\inputminted{c++}{counting_sort.cpp}
\subsection*{Результат виконання програми}
\inputminted{c++}{counting_sort_out}

\subsection*{Сортування бінарними вставками}
\inputminted{c++}{bin_ins_sort.cpp}
\subsection*{Результат виконання програми}
\inputminted{c++}{bin_ins_sort_out}

\section*{Висновок}

Під час виконання цієї лабораторної роботи я краще
зрозумів алгоритми сортування підрахунком, за розрядами,
бінарними вставками та алгоритм Шелла.

\section*{Відповіді на контрольні запитання}
\begin{itemize}
	\question Що таке алгоритм?
	\answer послідовність дій для виконання певного завдання.

	\question Що таке сортування?
	\answer  Упорядковування елементів деякої структури даних за певними правилами

	\question  Для чого використовуються алгоритми сортування?
	\answer для приведення даних у зручний для аналізу вигляд.

	\question Які основні характеристики алгоритмів сортування?
		\answer \begin{itemize}
				\item Стійкість
				\item природність поведінки
				\item часткова впорядкованість
				\item використання операції порівняння
			\end{itemize}
	\question Що таке ефективність алгоритму сортування?
	\answer Це оцінка того, наскільки швидко і ефективно алгоритм може відсортувати дані.

	\question Прямі та швидкі методи сортування.
	\answer Прямі методи сортування (також відомі як методи обміну) сортують масив шляхом порівнянь і обміну значень елементів, а швидкі зазвичай розділяють його на менші масиви і сортують їх окремо.

	\question  Що таке швидкодія алгоритму?
	\answer Міра продуктивності алгоритму, яка відображає, наскільки швидко він виконується
\end{itemize}

\end{document}
Між елементами списку повинні існувати відносини, що допускають повне упорядкування
(відношення порядку).
