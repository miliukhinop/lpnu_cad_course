%-shell-escape, якщо використовуєте minted
\documentclass[a4paper, 12pt, oneside]{extarticle}
\input{$HOME/Templates/lpnu_doc_templates/settings/preamble.tex}
% якщо домахуються за Times New Roman, то
% використовуєте xelatex і цей файл:
\input{$HOME/Templates/lpnu_doc_templates/settings/font_styles.tex}
\input{$HOME/Templates/lpnu_doc_templates/settings/minted_settings.tex}

\newcommand\Variant{4}
\newcommand\Date{\the\day.\the\month.\the\year}
\newcommand\Discipline{Алгоритмізація та програмування, частина 2}
\newcommand\Instructor{Кулешник Я. Ф.}

\newcommand\Type{\Lab}
\newcommand\Number{7}
\newcommand\Topic{Створення та ведення 2-3 та 2-3-4 дерев}

\begin{document}
\Margins

\input{$HOME/Templates/lpnu_doc_templates/parts/header.tex}
Ознайомитися iз способом подання даних в оперативнiй пам’ятi ЕОМ у
виглядi 2-3 і 2-3-4 дерев. Оволодiти методами роботи iз 2-3 та 2-3-4 деревами.
\section*{Індивідуальне завдання}

\paragraph{Варіант А) Програмно реалiзувати:}
\begin{enumerate}
\item процедуру вставляння нового елемента у 2-3 дерево;
\item процедуру видалення з 2-3 дерева;
\item процедуру пошуку елемента у 2-3 деревi;
\end{enumerate}

\subsection*{Завдання 1}

Нарисувати 2-3 дерево, що отримується послідовним введенням
наступних елементів: 7,9,10,2,20,15,50,120,54,63,23

\subsection*{Завдання 2}

Показати результат видалення елементів 20, 120, 23, 10 із
побудованого у п.1 2-3 дерева.

\section*{Етапи розв'язку}

Реалізував 2-3 дерево та записав у нього значення згідно з завданням,
потім видалив указані елементи.

\subsection*{Код програми}

\inputminted{c++}{code.cpp}

\subsection*{Результат виконання програми}
\inputminted{text}{out}

\section*{Висновок}

Під час виконання цієї лабораторної роботи я ознайомився з
утіленням 2-3 дерев засобами мови c++.

\section*{Відповіді на контрольні запитання}
\begin{itemize}
	\question Яка структура називається 2-3 деревом?
	\answer 2-3-деревом називається структура даних, яка є Б-деревом порядку 3, у
якій кожний вузол, що не є листом, має двох чи трьох нащадків, а довжини
всіх шляхів із кореня в листки однакові.
	\question Де розташовуються данi у 2-3 деревi?
	\answer У листових вузлах

	\question Як здiйснюється вставляння нового елемента у 2-3 дерево?
	\answer Після знаходження відповідного місця, елемент додається. Якщо в результаті вставки кількість ключів у вузлі перевищена, відбувається перерозподіл --- середній ключ передається батьківському вузлу, а ті, що зліва та справа формують окремі вузли.

	\question Як здiйснюється видалення з 2-3 дерева?
	\answer Видалення елемента з 2-3 дерева може бути трохи складнішим, ніж вставка, оскільки воно може призвести до порушення властивостей дерева. Існує кілька можливих випадків для видалення з 2-3 дерева:
		\begin{enumerate}
			\item Видалення елемента з листового вузла, якщо листовий вузол має більше одного елемента. Просто видаляємо елемент з вузла і перенумеровуємо всі наступні ключі в порядку зростання.

			\item Видалення елемента з листового вузла, якщо це єдиний елемент в вузлі. Видаляємо елемент з вузла і виконуємо операцію злиття з сусіднім вузлом, якщо цей сусідній вузол також є листовим вузлом і має більше одного елемента. Якщо немає жодного сусіднього вузла з більше ніж одним елементом, то ми зливаємо цей вузол зі своїм сусіднім вузлом і знищуємо порожній вузол.

			\item Видалення елемента з внутрішнього вузла. Знаходимо вузол, який містить наступний ключ після ключа, що ми хочемо видалити. Знаходимо найменший ключ в піддереві, корінь якого є цим наступним вузлом. Замінюємо ключ, який ми хочемо видалити, на знайдений ключ. Потім рекурсивно видаляємо замінений ключ з відповідного піддерева.

			\item Якщо після видалення вузол став порожнім, то ми виконуємо операцію злиття з сусіднім вузлом, якщо цей сусідній вузол має два елементи. Якщо обидва сусідні вузли є листовими і мають тільки один елемент, то ми зливаємо ці вузли разом, або видаляє
		\end{enumerate}

	\question Яка структура називається 2-4 деревом?
	\answer 2-3-4-деревом (також називається 2-4 дерево) називається структура
даних, яка є Б-деревом порядку 4, у якій кожний вузол, що не є листом, має
двох, трьох або чотирьох нащадків, а довжини всіх шляхів із кореня в листки
однакові. Порожнє дерево або дерево, що складається з 1 вузла, також є 2-4-
деревом.
	\question Де розташовуються данi у 2-4 деревi?
	\answer Так само в листових вузлах
	\question Як здiйснюється вставляння нового елемента у 2-4 дерево?
	\answer За схожим алгоритмом, як у 2-3 дерево --- якщо стається переповнення вузла, то його розділяють і виносять середній елемент у бітьківський вузол.
	\question Як здiйснюється видалення з 2-4 дерева?
	\answer
Якщо видаляється ключ з листової сторінки і кількість ключів в сторінці менше двох, то з іншої сторінки (якщо така існує) переноситься один ключ. Якщо такої сторінки немає, то дерево може зменшитись на рівень.

Якщо видаляється ключ з нелистової сторінки, то виконуються такі дії:

		\begin{enumerate}
			\item Якщо попередній ключ в піддереві має більше одного ключа, то він замінюється на максимальний ключ лівого піддерева або мінімальний ключ правого піддерева.

			\item Якщо попередній ключ має рівно один ключ і його сусід має більше двох ключів, то один ключ з його сусіда переноситься до поточного ключа.

			\item Якщо попередній ключ має рівно один ключ і його сусід також має рівно один ключ, то поточний ключ і його сусід об'єднуються у одну сторінку.
		\end{enumerate}
\end{itemize}

\end{document}
1.
2.
3.
4.
