%-shell-escape, якщо використовуєте minted
\documentclass[a4paper, 12pt, oneside]{extarticle}
\input{$HOME/Templates/lpnu_doc_templates/settings/preamble.tex}
% якщо домахуються за Times New Roman, то
% використовуєте xelatex і цей файл:
\input{$HOME/Templates/lpnu_doc_templates/settings/font_styles.tex}

\newcommand\Variant{12}
\newcommand\Date{12.05.\the\year}
\newcommand\Discipline{Комп'ютерна схемотехніка та архітектура комп'ютерних систем}
\newcommand\Instructor{Чкалов О. В.}

\newcommand\Type{\Lab}
\newcommand\Number{2}
\newcommand\Topic{Дослідження логічних елементів}
\graphicspath{{images/}}

\begin{document}
\Margins

\input{$HOME/Templates/lpnu_doc_templates/parts/header.tex}

дослідити роботу та принципи побудови логічних елементів
на резистивно-транзисторній логіці та МОН-транзисторах.

\section*{Індивідуальне завдання}

\subsection*{Завдання 1}

\begin{tabular}{c||c}
	12 & І-НЕ, НЕ, ЧИ-НЕ (РТЛ), ЧИ, І (МОН)
\end{tabular}

\subsection*{Завдання 2}

Побудувати чотири схеми в програмі NI Mutisim на логічних елементах та
використовуючи мікросхеми цих логічних елементів.
\paragraph{1} побудувати задану схему на логічних елементах і на
мікросхемах КМОН 4000 – серії, отримати логічну функцію по схемі та її
таблицю істинності використовуючи логічний перетворювач (Logic converter).
\paragraph{2} побудувати схему на логічних елементах і на мікросхемах
КМОН 4000 – серії по заданій логічній функції, отримати за логічною схемою
таблицю істинності досліджуваної логічної функції використовуючи логічний
перетворювач (Logic converter).

\begin{figure}[h]
	\includegraphics[width=\textwidth]{task_2}
	% \caption{І-НЕ}
\end{figure}

\section*{Етапи розв'язку}

\subsection*{Завдання 1}

\begin{figure}[h]
	\includegraphics[width=\textwidth]{NAND}
	\caption{І-НЕ}
\end{figure}
\begin{figure}[h]
	\includegraphics[width=\textwidth]{NOR}
	\caption{ЧИ-НЕ}
\end{figure}
\begin{figure}[h]
	\includegraphics[width=\textwidth]{NO}
	\caption{НЕ}
\end{figure}
\begin{figure}[h]
	\includegraphics[width=\textwidth]{fet_OR}
	\caption{АБО (МОН)}
\end{figure}
\begin{figure}[ht]
	\includegraphics[width=\textwidth]{fet_AND}
	\caption{І (МОН)}
\end{figure}

\clearpage

\subsection*{Завдання 2}

\begin{figure}[h]
	\includegraphics[width=\textwidth]{2.1}
	\caption{1 частина завдання}
\end{figure}
\begin{figure}[h]
	\includegraphics[width=\textwidth]{2.2}
	\caption{2 частина завдання}
\end{figure}

\clearpage

\section*{Висновок}

Дослідив роботу та принципи побудови логічних елементів на резистивно-транзисторній логіці та МОН-транзисторах,
побудував схеми логічних функцій.

\section*{Відповіді на контрольні запитання}
\begin{itemize}
	\question Які існують логічні операції?
		\answer Кон'юнкція, диз'юнкція, імплікація, рівносильність(еквівалентність), заперечення та інші.

	\question Що називається логічним елементом?
	\answer Логічний елемент - це електронний пристрій або компонент, який виконує логічні операції над сигналами логічних змінних

	\question Які логічні елементи реалізують логічні операції?
		\answer Заперечення реалізує інвертор (НЕ), диз'юнкцію --- елемент ЧИ, кон'юнкцію --- І, заперечення диз'юнкції --- НЕ ЧИ, зап. кон'юнкції --- НЕ І.

	\question Вказати область застосування логічних елементів.
	\answer Побудова логічних схем, мікропроцесорів, логічних вентилів.

	\question Якими методами описують логічні операції?
	\answer Логічні операції описують за допомогою:
		\begin{itemize}
	\item таблиць істинності
	\item булевих функцій
	\item графічного методу
	\item алгебраїчного методу
		\end{itemize}
\end{itemize}

\end{document}
