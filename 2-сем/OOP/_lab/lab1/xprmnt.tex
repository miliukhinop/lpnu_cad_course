%xelatex
\documentclass[a4paper, 12pt, oneside]{extarticle}
%shell-escape
\input{../../preamble.tex}

\newcommand\Variant{4}
\newcommand\Date{03.03.\the\year}
\newcommand\Discipline{Об'єктно-Орієнтоване програмування}
\newcommand\Instructor{Патерега Ю.І.}

\newcommand\Lab{лабораторної роботи}
\newcommand\Pract{практичної роботи}
\newcommand\Work{\Lab~\No1}
\newcommand\Theme{Знайомство та налаштування середовища Microsoft Visual Studio 2019}

\begin{document}
\Margins

\input{../../header.tex}

Ознайомитись та налаштувати програмне середовище Microsoft Visual
Studio 2019.

$\delta~\pi~\chi~\xi$
$x + y = 21$

\subsubsection*{Завдання 1}

Виконання основних операцій в інтегрованому середовищі Visual Studio
(консольний режим C++).

\subsubsection*{Завдання 2}

На диску D: створити папку lab1, де буде зберігатися проект цього завдання.
Набрати текст програми (з коментарями кожного рядка коду), яка видає
привітання:
\texttt{/* Прізвище Ім’я По батькові, група Тх-хх, варіант № хх * Лабораторна
робота № 1. Знайомство та налаштування програмного середовища}

%---------------------------------------------
\section*{Етапи розв'язку}

\subsection*{Код програми}

\listinginput{1}{./1.cpp}
\verbatiminput{1.cpp}
\lstinputlisting[frame=single, firstline=1]{./1.cpp}
%\VerbatimInput[defineactive=\def^^L{\aftergroup\aftereject}]{1.cpp}
\VerbatimInput{1.cpp}
\inputminted{c++}{1.cpp}

\subsection*{Вміст тестового файлу}
\verbatiminput{./test1.sh}

\subsection*{Результат виконання програми}

\begin{verbatim}
Прізвище:
Ім'я:
По батькові:
Група:
Варіант:
Мілюхін Олександр Павлович, група ПП-14, варіант №4, Лабораторна робота №1.
Знайомство та налаштування програмного середовища Microsoft Visual Studio
\end{verbatim}

\section*{Висновок}
Програма скомпілювалася й виконалась успішно, правильно
інтерпретувала введені з тестового файлу дані й вивела їх.

\section*{Відповіді на контрольні запитання}
\begin{enumerate}
\item Що таке машинні та високорівневі мови програмування? Порівняти відомі вам мови.

Машинна мова --- набір команд, які виконуються безпосередньо центральним процесором
		комп'ютера без транслятора. Високорівнева мова використовує абстракції
		різних видів для полегшення розуміння програмного коду.

\item Визначити та описати парадигми програмування.

Парадигма --- підхід до осмислення явища. Тобто у програмуванні це різні погляди
		на вирішення одної задачі. Існує багато парадигм програмування, зокрема структурне(сумнівно),
об'єктно-орієнтоване, імперативне, паралельне.

\item Що таке системне програмування та його застосування?
	Системне програмування --- розробка програм, які працюють із системним ПЗ або
	апаратним забезпеченням комп'ютера. Для системних програм швидкість є набагато
	важливішою, ніж для прикладних, адже вони обслуговують усі інші програми.

\item Дати визначення структурному та процедурному програмуванню.

Структурне програмування --- конструювання програм, що використовує лише ієрархічно вкладені конструкції,кожна з яких має єдину точку входу та єдину точку виходу. Передбачає:

		\begin{itemize}
\item	Послідовне виконання - виконання певної операції в тому порядку, в якому вона записана в тексті самої програми;
\item	Розгалуження - це одна з двох або більше операцій;
\item	Цикл - повторення однієї і тієї ж операції поки виконується конкретно задана умова.
		\end{itemize}

Процедурне програмування --- парадигма програмування, заснована на концепції виклику процедури.

\item Що таке модульне та об’єктно-орієнтоване програмування? Порівняти

Модульне програмування --- парадигма програмування, орієнтована на зменшення складності програм та можливості перенесення окремих рішень з одних програмних проектів у інші. Побудована як наслідок модульна архітектура підкреслює поділ функціональності програми на незалежні змінні модулі, таких, що кожен з них містить усе необхідне, щоб виконати тільки один аспект необхідної функціональності. Модулі, як правило, включені в програму через інтерфейси.

ООП --- одна з парадигм програмування, яка розглядає програму як множину «об'єктів», що взаємодіють між собою. Основу ООП складають чотири основні концепції: інкапсуляція, успадкування, поліморфізм та абстракція
\end{enumerate}

\end{document}
