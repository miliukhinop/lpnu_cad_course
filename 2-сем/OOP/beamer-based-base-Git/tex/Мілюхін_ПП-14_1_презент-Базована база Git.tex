\documentclass[aspectratio=169]{beamer}

\input{/home/sasha/Templates/lpnu_doc_templates/settings/preamble.tex}

\usepackage[utf8]{inputenc}
\usepackage[T2A]{fontenc}
\usepackage[ukrainian]{babel}
\usepackage{wrapfig}

%\usepackage{subfiles}
%\setbeamertemplate{caption}[numbered]

\usetheme{Malmoe} % Copenhagen
%\usecolortheme{}

%\usefonttheme{serif}
\setbeamerfont{title}{series=\bfseries,parent=structure}
\setbeamerfont{frametitle}{series=\bfseries,parent=structure}

\title{Базована база Git}
\author{\Lname~\Fname}
%\date{}

\institute{Національний університет ``Львівська Політехніка''\\[\medskipamount]
\includegraphics[height=.4\textheight]{$HOME/Templates/lpnu_doc_templates/lpnu_logo.png}
}

\begin{document}
\maketitle

\tableofcontents

\section{Тупий відстежувач вмісту}

\begin{frame}{Git --- тупий відстежувач вмісту}{Або розподілена система контролю версій}

На початках (1991–2002) зміни в ядро Linux передавалися як патчі та архіви. У 2002 р. проєкт
почав використовувати пропрієтарну РСКВ BitKeeper.
	\textbf {Git народилася у 2005, коли чувакам набридло миритися з комерційною натурою BitKeeper.}

	%\begin{columns}
	%	\column{.25\textwidth}
	%	\begin{figure}
	%		%\includegraphics[width=.9\textwidth]{<++>}
	%		\caption{<++>}
	%	\end{figure}
	%		%\includegraphics[height=.5\textheight]{<++>}
	%	\column{.75\textwidth}
	%\end{columns}

\end{frame}


\section{Вчитися}

\subsection{CLI}

\begin{frame}{Стиль}{CLI --- імба}
	\begin{block}{Як не помилятися}
		\begin{itemize}
			\item Натискати \texttt{Tab} для автодоповнення
			\item Стрілки вгору-вниз для переміщення по історії команд
		\end{itemize}
	\end{block}
\end{frame}

\begin{frame}[fragile]{Швидкий копен-дупен}

	\begin{verbatim}
			git init
			git add .
			git commit -m "початок"
	\end{verbatim}

	\begin{verbatim}
		git reset --hard
	\end{verbatim}

	\begin{verbatim}
		git commit -a -m "якісь зміни"
	\end{verbatim}

\end{frame}

\subsection{Visual Studio}

\section{Що далі?}

\begin{frame}{Шткуи}

https://ohshitgit.com/

\end{frame}

\end{document}
