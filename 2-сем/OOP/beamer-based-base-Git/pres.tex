% Options for packages loaded elsewhere
\PassOptionsToPackage{unicode}{hyperref}
\PassOptionsToPackage{hyphens}{url}
\PassOptionsToPackage{dvipsnames,svgnames,x11names}{xcolor}
%
\documentclass[
  ignorenonframetext,
  aspectratio=169,
]{beamer}
\usepackage{pgfpages}
\setbeamertemplate{caption}[numbered]
\setbeamertemplate{caption label separator}{: }
\setbeamercolor{caption name}{fg=normal text.fg}
\beamertemplatenavigationsymbolsempty
% Prevent slide breaks in the middle of a paragraph
\widowpenalties 1 10000
\raggedbottom
\setbeamertemplate{part page}{
  \centering
  \begin{beamercolorbox}[sep=16pt,center]{part title}
    \usebeamerfont{part title}\insertpart\par
  \end{beamercolorbox}
}
\setbeamertemplate{section page}{
  \centering
  \begin{beamercolorbox}[sep=12pt,center]{part title}
    \usebeamerfont{section title}\insertsection\par
  \end{beamercolorbox}
}
\setbeamertemplate{subsection page}{
  \centering
  \begin{beamercolorbox}[sep=8pt,center]{part title}
    \usebeamerfont{subsection title}\insertsubsection\par
  \end{beamercolorbox}
}
\AtBeginPart{
  \frame{\partpage}
}
\AtBeginSection{
  \ifbibliography
  \else
    \frame{\sectionpage}
  \fi
}
\AtBeginSubsection{
  \frame{\subsectionpage}
}
\usepackage{amsmath,amssymb}
\usepackage{iftex}
\ifPDFTeX
  \usepackage[T2A]{fontenc}
  \usepackage[utf8]{inputenc}
  \usepackage{textcomp} % provide euro and other symbols
\else % if luatex or xetex
  \usepackage{unicode-math} % this also loads fontspec
  \defaultfontfeatures{Scale=MatchLowercase}
  \defaultfontfeatures[\rmfamily]{Ligatures=TeX,Scale=1}
\fi
\usetheme[]{Malmoe}
\usefonttheme{serif} % use mainfont rather than sansfont for slide text
\ifPDFTeX\else
  % xetex/luatex font selection
  \setmainfont[]{Fira Sans}
  \setmonofont[]{Fantasque Sans Mono}
\fi
% Use upquote if available, for straight quotes in verbatim environments
\IfFileExists{upquote.sty}{\usepackage{upquote}}{}
\IfFileExists{microtype.sty}{% use microtype if available
  \usepackage[]{microtype}
  \UseMicrotypeSet[protrusion]{basicmath} % disable protrusion for tt fonts
}{}
\makeatletter
\@ifundefined{KOMAClassName}{% if non-KOMA class
  \IfFileExists{parskip.sty}{%
    \usepackage{parskip}
  }{% else
    \setlength{\parindent}{0pt}
    \setlength{\parskip}{6pt plus 2pt minus 1pt}}
}{% if KOMA class
  \KOMAoptions{parskip=half}}
\makeatother
\usepackage{xcolor}
\newif\ifbibliography
\usepackage{color}
\usepackage{fancyvrb}
\newcommand{\VerbBar}{|}
\newcommand{\VERB}{\Verb[commandchars=\\\{\}]}
\DefineVerbatimEnvironment{Highlighting}{Verbatim}{commandchars=\\\{\}}
% Add ',fontsize=\small' for more characters per line
\newenvironment{Shaded}{}{}
\newcommand{\AlertTok}[1]{\textcolor[rgb]{1.00,0.00,0.00}{\textbf{#1}}}
\newcommand{\AnnotationTok}[1]{\textcolor[rgb]{0.38,0.63,0.69}{\textbf{\textit{#1}}}}
\newcommand{\AttributeTok}[1]{\textcolor[rgb]{0.49,0.56,0.16}{#1}}
\newcommand{\BaseNTok}[1]{\textcolor[rgb]{0.25,0.63,0.44}{#1}}
\newcommand{\BuiltInTok}[1]{\textcolor[rgb]{0.00,0.50,0.00}{#1}}
\newcommand{\CharTok}[1]{\textcolor[rgb]{0.25,0.44,0.63}{#1}}
\newcommand{\CommentTok}[1]{\textcolor[rgb]{0.38,0.63,0.69}{\textit{#1}}}
\newcommand{\CommentVarTok}[1]{\textcolor[rgb]{0.38,0.63,0.69}{\textbf{\textit{#1}}}}
\newcommand{\ConstantTok}[1]{\textcolor[rgb]{0.53,0.00,0.00}{#1}}
\newcommand{\ControlFlowTok}[1]{\textcolor[rgb]{0.00,0.44,0.13}{\textbf{#1}}}
\newcommand{\DataTypeTok}[1]{\textcolor[rgb]{0.56,0.13,0.00}{#1}}
\newcommand{\DecValTok}[1]{\textcolor[rgb]{0.25,0.63,0.44}{#1}}
\newcommand{\DocumentationTok}[1]{\textcolor[rgb]{0.73,0.13,0.13}{\textit{#1}}}
\newcommand{\ErrorTok}[1]{\textcolor[rgb]{1.00,0.00,0.00}{\textbf{#1}}}
\newcommand{\ExtensionTok}[1]{#1}
\newcommand{\FloatTok}[1]{\textcolor[rgb]{0.25,0.63,0.44}{#1}}
\newcommand{\FunctionTok}[1]{\textcolor[rgb]{0.02,0.16,0.49}{#1}}
\newcommand{\ImportTok}[1]{\textcolor[rgb]{0.00,0.50,0.00}{\textbf{#1}}}
\newcommand{\InformationTok}[1]{\textcolor[rgb]{0.38,0.63,0.69}{\textbf{\textit{#1}}}}
\newcommand{\KeywordTok}[1]{\textcolor[rgb]{0.00,0.44,0.13}{\textbf{#1}}}
\newcommand{\NormalTok}[1]{#1}
\newcommand{\OperatorTok}[1]{\textcolor[rgb]{0.40,0.40,0.40}{#1}}
\newcommand{\OtherTok}[1]{\textcolor[rgb]{0.00,0.44,0.13}{#1}}
\newcommand{\PreprocessorTok}[1]{\textcolor[rgb]{0.74,0.48,0.00}{#1}}
\newcommand{\RegionMarkerTok}[1]{#1}
\newcommand{\SpecialCharTok}[1]{\textcolor[rgb]{0.25,0.44,0.63}{#1}}
\newcommand{\SpecialStringTok}[1]{\textcolor[rgb]{0.73,0.40,0.53}{#1}}
\newcommand{\StringTok}[1]{\textcolor[rgb]{0.25,0.44,0.63}{#1}}
\newcommand{\VariableTok}[1]{\textcolor[rgb]{0.10,0.09,0.49}{#1}}
\newcommand{\VerbatimStringTok}[1]{\textcolor[rgb]{0.25,0.44,0.63}{#1}}
\newcommand{\WarningTok}[1]{\textcolor[rgb]{0.38,0.63,0.69}{\textbf{\textit{#1}}}}
\setlength{\emergencystretch}{3em} % prevent overfull lines
\providecommand{\tightlist}{%
  \setlength{\itemsep}{0pt}\setlength{\parskip}{0pt}}
\setcounter{secnumdepth}{-\maxdimen} % remove section numbering
\ifLuaTeX
\usepackage[bidi=basic]{babel}
\else
\usepackage[bidi=default]{babel}
\fi
\babelprovide[main,import]{ukrainian}
\ifPDFTeX
\else
\babelfont[ukrainian]{rm}{Fira Sans}
\fi
% get rid of language-specific shorthands (see #6817):
\let\LanguageShortHands\languageshorthands
\def\languageshorthands#1{}
\ifLuaTeX
  \usepackage{selnolig}  % disable illegal ligatures
\fi
\IfFileExists{bookmark.sty}{\usepackage{bookmark}}{\usepackage{hyperref}}
\IfFileExists{xurl.sty}{\usepackage{xurl}}{} % add URL line breaks if available
\urlstyle{same}
\hypersetup{
  pdftitle={Базована база Git},
  pdfauthor={Олександр Мілюхін},
  pdflang={uk-UA},
  colorlinks=true,
  linkcolor={white},
  filecolor={Maroon},
  citecolor={Blue},
  urlcolor={blue},
  pdfcreator={LaTeX via pandoc}}

\title{Базована база Git}
\author{Олександр Мілюхін}
\date{}
\institute{Національний університет ``Львівська Політехніка''}

\begin{document}
\frame{\titlepage}

\begin{frame}{План}
\protect\hypertarget{ux43fux43bux430ux43d}{}
\tableofcontents
\end{frame}

\hypertarget{ux442ux443ux43fux438ux439-ux432ux456ux434ux441ux442ux435ux436ux443ux432ux430ux447-ux432ux43cux456ux441ux442ux443}{%
\section{Тупий відстежувач
вмісту}\label{ux442ux443ux43fux438ux439-ux432ux456ux434ux441ux442ux435ux436ux443ux432ux430ux447-ux432ux43cux456ux441ux442ux443}}

\begin{frame}{Git --- тупий відстежувач вмісту}
\protect\hypertarget{git-ux442ux443ux43fux438ux439-ux432ux456ux434ux441ux442ux435ux436ux443ux432ux430ux447-ux432ux43cux456ux441ux442ux443}{}
На початках (1991--2002) зміни в ядро Linux передавалися як патчі та
архіви. У 2002 р. проєкт почав використовувати пропрієтарну РСКВ
BitKeeper.

\textbf{Git народилася у 2005, коли чувакам набридло миритися з
комерційною натурою BitKeeper.}

\begin{block}{Чому \textbf{розподілена}?}
\protect\hypertarget{ux447ux43eux43cux443-ux440ux43eux437ux43fux43eux434ux456ux43bux435ux43dux430}{}
\textless++\textgreater{} бо, на відміну від старих централізованих
систем, розподілена система дає всім, хто завантажив репозиторій,
можливість перейти до будь-якого збереженого стану без звертання до
сервера.
\end{block}
\end{frame}

\begin{frame}{Що можна відстежувати?}
\protect\hypertarget{ux449ux43e-ux43cux43eux436ux43dux430-ux432ux456ux434ux441ux442ux435ux436ux443ux432ux430ux442ux438}{}
Все, що завгодно. Контроль версій може вам знадобитися в місцях, де ви
його найменше сподіваєтеся.

\begin{itemize}
\tightlist
\item
  програми
\item
  конфігураційні файли
\item
  документи (також вебсторінки)
\item
  придумайте
\end{itemize}

Зміни в цій презентації також контрольовані за допомогою Git.
\end{frame}

\hypertarget{ux43eux441ux43dux43eux432ux438-ux432ux436ux438ux442ux43aux443}{%
\section{Основи
вжитку}\label{ux43eux441ux43dux43eux432ux438-ux432ux436ux438ux442ux43aux443}}

\hypertarget{cli}{%
\subsection{CLI}\label{cli}}

\begin{frame}[fragile]{CLI --- імба}
\protect\hypertarget{cli-ux456ux43cux431ux430}{}
Переваги використання Git через інтерфейс командного рядка:

\pause

\begin{itemize}
\tightlist
\item
  Ви не будете прив'язані до інтерфейсу якогось IDE \pause
\item
  Краще зрозумієте програму \pause
\item
  швидше будете знаходити рішення різних проблем \pause
\item
  будете виглядати як крутий гакер \pause
\end{itemize}

\begin{block}{Як не помилятися та швидше рухатись}
\protect\hypertarget{ux44fux43a-ux43dux435-ux43fux43eux43cux438ux43bux44fux442ux438ux441ux44f-ux442ux430-ux448ux432ux438ux434ux448ux435-ux440ux443ux445ux430ux442ux438ux441ux44c}{}
\pause

\begin{itemize}
\tightlist
\item
  Натискати \textbf{\texttt{Tab}} (часом кілька разів) для
  автодоповнення \pause
\item
  \textbf{Стрілки} вгору-вниз для переміщення по історії команд \pause
\item
  \emph{не вчити} команди, а \textbf{використовувати} їх
\end{itemize}
\end{block}
\end{frame}

\begin{frame}[fragile]{Швидкий копен-дупен}
\protect\hypertarget{ux448ux432ux438ux434ux43aux438ux439-ux43aux43eux43fux435ux43d-ux434ux443ux43fux435ux43d}{}
\begin{block}{Ініціалізація репо:}
\protect\hypertarget{ux456ux43dux456ux446ux456ux430ux43bux456ux437ux430ux446ux456ux44f-ux440ux435ux43fux43e}{}
\begin{Shaded}
\begin{Highlighting}[]
\FunctionTok{git}\NormalTok{ init}
\FunctionTok{git}\NormalTok{ add .}
\FunctionTok{git}\NormalTok{ commit }\AttributeTok{{-}m} \StringTok{"мяу"}
\end{Highlighting}
\end{Shaded}

\pause
\end{block}

\begin{block}{Повернення до початкового стану:}
\protect\hypertarget{ux43fux43eux432ux435ux440ux43dux435ux43dux43dux44f-ux434ux43e-ux43fux43eux447ux430ux442ux43aux43eux432ux43eux433ux43e-ux441ux442ux430ux43dux443}{}
\begin{Shaded}
\begin{Highlighting}[]
\FunctionTok{git}\NormalTok{ reset }\AttributeTok{{-}{-}hard}
\end{Highlighting}
\end{Shaded}

\pause
\end{block}

\begin{block}{Внесення нових змін:}
\protect\hypertarget{ux432ux43dux435ux441ux435ux43dux43dux44f-ux43dux43eux432ux438ux445-ux437ux43cux456ux43d}{}
\begin{Shaded}
\begin{Highlighting}[]
\FunctionTok{git}\NormalTok{ commit }\AttributeTok{{-}a} \AttributeTok{{-}m} \StringTok{"якісь зміни"}
\end{Highlighting}
\end{Shaded}

\begin{quote}
-a каже додати до індексу всі зміни/видалення
\end{quote}
\end{block}
\end{frame}

\begin{frame}{bruh}
\protect\hypertarget{bruh}{}
\begin{columns}
\column{.5\textwidth}
\begin{figure}
    \includegraphics[width=.7\textwidth]{images/image_proxy.jpg}
\end{figure}
\end{columns}
\end{frame}

\begin{frame}[fragile]{База}
\protect\hypertarget{ux431ux430ux437ux430}{}
\begin{block}{Додання}
\protect\hypertarget{ux434ux43eux434ux430ux43dux43dux44f}{}
\begin{Shaded}
\begin{Highlighting}[]
\FunctionTok{git}\NormalTok{ add }\PreprocessorTok{[}\SpecialStringTok{назва}\PreprocessorTok{{-}}\SpecialStringTok{файлу}\PreprocessorTok{]}   \CommentTok{\# додати файл}
\end{Highlighting}
\end{Shaded}

\pause
\end{block}

\begin{block}{Видалення}
\protect\hypertarget{ux432ux438ux434ux430ux43bux435ux43dux43dux44f}{}
видалити файл і з індексу, і з середовища \textless++\textgreater.
\texttt{-r}, щоб рекурсивно:

\begin{Shaded}
\begin{Highlighting}[]
\FunctionTok{git}\NormalTok{ rm }\PreprocessorTok{[}\SpecialStringTok{назва}\PreprocessorTok{{-}}\SpecialStringTok{файлу}\PreprocessorTok{]}        \CommentTok{\# (remove)}
\FunctionTok{git}\NormalTok{ rm }\AttributeTok{{-}{-}cached}\NormalTok{ назва{-}файлу }\CommentTok{\# видалити файл тільки з індексу}
\end{Highlighting}
\end{Shaded}

\pause
\end{block}

\begin{block}{Перейменування}
\protect\hypertarget{ux43fux435ux440ux435ux439ux43cux435ux43dux443ux432ux430ux43dux43dux44f}{}
\begin{Shaded}
\begin{Highlighting}[]
\FunctionTok{git}\NormalTok{ mv bug.c feature.c  }\CommentTok{\# фактично переміщення (move)}
\end{Highlighting}
\end{Shaded}
\end{block}
\end{frame}

\begin{frame}[fragile]{Більше команд для відстежування}
\protect\hypertarget{ux431ux456ux43bux44cux448ux435-ux43aux43eux43cux430ux43dux434-ux434ux43bux44f-ux432ux456ux434ux441ux442ux435ux436ux443ux432ux430ux43dux43dux44f}{}
\begin{block}{Історія}
\protect\hypertarget{ux456ux441ux442ux43eux440ux456ux44f}{}
\begin{Shaded}
\begin{Highlighting}[]
\FunctionTok{git}\NormalTok{ log     }\CommentTok{\# покаже список останніх комітів і їхні хеші SHA1}
\FunctionTok{git}\NormalTok{ reflog  }\CommentTok{\# покаже зміни в гілках та посиланнях \textless{}++\textgreater{}}
\end{Highlighting}
\end{Shaded}

\pause
\end{block}

\begin{block}{Як просто повернутися?}
\protect\hypertarget{ux44fux43a-ux43fux440ux43eux441ux442ux43e-ux43fux43eux432ux435ux440ux43dux443ux442ux438ux441ux44f}{}
Повернутися до потрібного коміту, створивши нову гілку:

\begin{Shaded}
\begin{Highlighting}[]
\FunctionTok{git}\NormalTok{ checkout }\PreprocessorTok{[}\SpecialStringTok{хеш}\PreprocessorTok{{-}}\SpecialStringTok{коміту}\PreprocessorTok{]} \PreprocessorTok{[}\SpecialStringTok{додатково}\PreprocessorTok{{-}}\SpecialStringTok{перелік}\PreprocessorTok{{-}}\SpecialStringTok{файлів}\PreprocessorTok{]}
\end{Highlighting}
\end{Shaded}

\begin{Shaded}
\begin{Highlighting}[]
\FunctionTok{git}\NormalTok{ checkout }\PreprocessorTok{[}\SpecialStringTok{назва}\PreprocessorTok{{-}}\SpecialStringTok{гілки}\PreprocessorTok{]} \CommentTok{\# перейде на вказану гілку}
\end{Highlighting}
\end{Shaded}

\pause
\end{block}

\begin{block}{Шо ви наробили коммітом? \textless++\textgreater{}}
\protect\hypertarget{ux448ux43e-ux432ux438-ux43dux430ux440ux43eux431ux438ux43bux438-ux43aux43eux43cux43cux456ux442ux43eux43c}{}
\begin{Shaded}
\begin{Highlighting}[]
\FunctionTok{git}\NormalTok{ diff}
\end{Highlighting}
\end{Shaded}
\end{block}
\end{frame}

\hypertarget{ux441ux435ux440ux432ux435ux440ux438-git}{%
\subsection{Сервери Git}\label{ux441ux435ux440ux432ux435ux440ux438-git}}

\begin{frame}{Самому сумно}
\protect\hypertarget{ux441ux430ux43cux43eux43cux443-ux441ux443ux43cux43dux43e}{}
Як працювати з іншими людьми? \textbf{Нам потрібен сервер!}

Ось кілька сервісів, що їх надають:

\begin{itemize}
\tightlist
\item
  \url{https://notabug.org/}
\item
  \url{https://github.com/}
\item
  \url{https://gitlab.com}
\item
  \url{https://sourceforge.net/}
\end{itemize}

\pause

\begin{block}{Або свій}
\protect\hypertarget{ux430ux431ux43e-ux441ux432ux456ux439}{}
Власний сервер Git запустити дуже просто, наприклад, повторюючи команди
з \href{https://landchad.net/git/}{цього посібника}. Для вебінтерфейсу
можна використати \textbf{cgit} або \textbf{gitea}, про які там теж
написано
\end{block}
\end{frame}

\begin{frame}[fragile]{Під'єднання віддаленого сховища}
\protect\hypertarget{ux43fux456ux434ux454ux434ux43dux430ux43dux43dux44f-ux432ux456ux434ux434ux430ux43bux435ux43dux43eux433ux43e-ux441ux445ux43eux432ux438ux449ux430}{}
\begin{block}{Щоб пов'язати локальний репозиторій із сервером:}
\protect\hypertarget{ux449ux43eux431-ux43fux43eux432ux44fux437ux430ux442ux438-ux43bux43eux43aux430ux43bux44cux43dux438ux439-ux440ux435ux43fux43eux437ux438ux442ux43eux440ux456ux439-ux456ux437-ux441ux435ux440ux432ux435ux440ux43eux43c}{}
\begin{Shaded}
\begin{Highlighting}[]
\FunctionTok{git}\NormalTok{ remote add }\PreprocessorTok{[}\SpecialStringTok{довільна}\PreprocessorTok{{-}}\SpecialStringTok{назва\_звич.}\PreprocessorTok{{-}}\SpecialStringTok{origin}\PreprocessorTok{]} \PreprocessorTok{[}\SpecialStringTok{ssh}\PreprocessorTok{{-}}\SpecialStringTok{або}\PreprocessorTok{{-}}\SpecialStringTok{https}\PreprocessorTok{{-}}\SpecialStringTok{адреса}\PreprocessorTok{]}
\end{Highlighting}
\end{Shaded}

\textbf{\emph{Рекомендую ssh.}}

\pause
\end{block}

\begin{block}{Завантажити репо}
\protect\hypertarget{ux437ux430ux432ux430ux43dux442ux430ux436ux438ux442ux438-ux440ux435ux43fux43e}{}
\begin{Shaded}
\begin{Highlighting}[]
\FunctionTok{git}\NormalTok{ clone }\PreprocessorTok{[}\SpecialStringTok{ssh}\PreprocessorTok{{-}}\SpecialStringTok{або}\PreprocessorTok{{-}}\SpecialStringTok{https}\PreprocessorTok{{-}}\SpecialStringTok{адреса}\PreprocessorTok{]}
\end{Highlighting}
\end{Shaded}

\pause
\end{block}

\begin{block}{Завантажити зміни}
\protect\hypertarget{ux437ux430ux432ux430ux43dux442ux430ux436ux438ux442ux438-ux437ux43cux456ux43dux438}{}
\begin{Shaded}
\begin{Highlighting}[]
\FunctionTok{git}\NormalTok{ fetch }\PreprocessorTok{[}\SpecialStringTok{назва}\PreprocessorTok{{-}}\SpecialStringTok{джерела}\PreprocessorTok{]} \PreprocessorTok{[}\SpecialStringTok{назва}\PreprocessorTok{{-}}\SpecialStringTok{гілки}\PreprocessorTok{]} \CommentTok{\# просто інформація}
\FunctionTok{git}\NormalTok{ pull }\PreprocessorTok{[}\SpecialStringTok{назва}\PreprocessorTok{{-}}\SpecialStringTok{джерела}\PreprocessorTok{]} \PreprocessorTok{[}\SpecialStringTok{назва}\PreprocessorTok{{-}}\SpecialStringTok{гілки}\PreprocessorTok{]}  \CommentTok{\# внесе локальні зміни}
\end{Highlighting}
\end{Shaded}
\end{block}
\end{frame}

\hypertarget{git-ux443-visual-studio}{%
\subsection{Git у Visual Studio}\label{git-ux443-visual-studio}}

\begin{frame}{Час лізти в комп'ютер}
\protect\hypertarget{ux447ux430ux441-ux43bux456ux437ux442ux438-ux432-ux43aux43eux43cux43fux44eux442ux435ux440}{}
\begin{columns}
\column{.6\textwidth}
\begin{figure}
    \includegraphics[width=\textwidth]{images/cat.jpg}
\end{figure}
\end{columns}
\end{frame}

\hypertarget{ux449ux43e-ux434ux430ux43bux456}{%
\section{Що далі?}\label{ux449ux43e-ux434ux430ux43bux456}}

\begin{frame}{Веб}
\protect\hypertarget{ux432ux435ux431}{}
\begin{itemize}
\tightlist
\item
  \url{https://ohshitgit.com/} --- мінімалістичний сайт із рішеннями
  поширених проблем (\url{https://dangitgit.com/} без матюків)
\item
  !! \url{https://git-scm.com/docs} --- Документація на сайті Git
\item
  \url{https://ndpsoftware.com/git-cheatsheet.html\#loc=index;} ---
  графічна шпаргалка
\item
  \url{https://training.github.com/downloads/ua/github-git-cheat-sheet/}
  --- ше шпаргалка
\item
  \url{http://www-cs-students.stanford.edu/~blynn/gitmagic/} --- гарна
  книжка з помилками в перекладі
\item
  \url{https://www.youtube.com/watch?v=mJ-qvsxPHpY} --- норм відео
\end{itemize}
\end{frame}

\begin{frame}{Більше книжок}
\protect\hypertarget{ux431ux456ux43bux44cux448ux435-ux43aux43dux438ux436ux43eux43a}{}
\begin{columns}
    \column{.65\textwidth}
    \begin{enumerate}
        \item \textbf{Ben Lynn.} Магія Git
        \item \textbf{Scott Chacon, Ben Straub.} Pro Git
        \item \textbf{Ryan Hodson.} git succinctly
    \end{enumerate}
    \column{.35\textwidth}
    \begin{figure}
        \includegraphics[width=.6\textwidth]{images/image_proxy.png}
        \caption{книжка про Git}
    \end{figure}
\end{columns}

\begin{block}{Або man-сторінки}
\protect\hypertarget{ux430ux431ux43e-man-ux441ux442ux43eux440ux456ux43dux43aux438}{}
зокрема gittutorial, giteveryday
\end{block}
\end{frame}

\end{document}
