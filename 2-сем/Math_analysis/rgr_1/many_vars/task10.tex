\documentclass[../rgr1.tex]{subfiles}

\begin{document}
% перевірити чи подвійна похідна норм
\Problem{Знайти екстремуми функції
$
	z = x^3y^2(6-x-y)
$
}
\Solution

\begin{dmath}
	\frac{\partial z}{\partial x}
	= y^2\Big(3x^2(6-x-y) + (0-1-0)\cdot x^3\Big)
	= y^2(18x^2 -3x^3 - 3x^2y -x^3)
	= -x^2y^2(4x+3y-18)
\end{dmath}

\begin{dmath}
	\frac{\partial z}{\partial y}
	= x^3 \Big( 2y(6-x-y) + (0-0-1)\cdot y^2 \Big)
	= x^3 (12y-2xy-2y^2-y^2)
	= -x^3y(3y+2x-12)
\end{dmath}
% ${\displaystyle \mathbf {H}_{f}={\begin{bmatrix}{\dfrac {\partial ^{2}f}{\partial x_{1}^{2}}}&{\dfrac {\partial ^{2}f}{\partial x_{1}\,\partial x_{2}}}&\cdots &{\dfrac {\partial ^{2}f}{\partial x_{1}\,\partial x_{n}}}\\[2.2ex]{\dfrac {\partial ^{2}f}{\partial x_{2}\,\partial x_{1}}}&{\dfrac {\partial ^{2}f}{\partial x_{2}^{2}}}&\cdots &{\dfrac {\partial ^{2}f}{\partial x_{2}\,\partial x_{n}}}\\[2.2ex]\vdots &\vdots &\ddots &\vdots \\[2.2ex]{\dfrac {\partial ^{2}f}{\partial x_{n}\,\partial x_{1}}}&{\dfrac {\partial ^{2}f}{\partial x_{n}\,\partial x_{2}}}&\cdots &{\dfrac {\partial ^{2}f}{\partial x_{n}^{2}}}\end{bmatrix}}}$
\begin{dmath}
	\begin{cases}
		-x^2y^2(4x+3y-18) = 0 \\
		-x^3y(3y+2x-12) = 0
	\end{cases} \implies
	\fbox{$x_1=y_1=0$} \implies
	\begin{cases}
		\left[
		\begin{aligned}
			-x^2y^2 = 0 \\
			(4x+3y-18) = 0
		\end{aligned} \right. \\
\\
		\left[
		\begin{aligned}
			-x^3y = 0 \\
			(3y+2x-12) = 0
		\end{aligned} \right. \\
	\end{cases} \implies
	-\begin{cases}
		4x+3y-18 = 0 \\
		3y+2x-12 = 0
	\end{cases} \implies
	\begin{cases}
		x_2=3 \\
		y_2=2
	\end{cases}
\end{dmath}

\begin{equation}
	\mathbf{G}(x,y) =
	\begin{bmatrix}
		\frac{\partial^2 z}{\partial x^2} & \frac{\partial^2 z}{\partial x\partial y} \\
		\frac{\partial^2 z}{\partial y\partial x} & \frac{\partial^2 z}{\partial y}
	\end{bmatrix}
	% \left(
	% \right)
\end{equation}
\begin{align}
	\frac{\partial^2 z}{\partial x^2}
	= \left(36\,{y}^{2}-6\,{y}^{3}\right)\,x-12\,{y}^{2}\,{x}^{2} \\
	\frac{\partial^2 z}{\partial y^2}
	= -6\,{x}^{3}\,y-2\,{x}^{4}+12\,{x}^{3} \\
	\frac{\partial^2 z}{\partial x\partial y} =
	\left(36\,y-9\,{y}^{2}\right)\,{x}^{2}-8\,y\,{x}^{3}
\end{align}
$$
A_1(0,0), A_2(0,2), A_3(3,0), A_4(3,2)
$$

\begin{align}
	\mathbf{G}(z) =
	\begin{bmatrix}
		\left(36\,{y}^{2}-6\,{y}^{3}\right)\,x-12\,{y}^{2}\,{x}^{2} & 	\left(36\,y-9\,{y}^{2}\right)\,{x}^{2}-8\,y\,{x}^{3} \\
		\left(36\,y-9\,{y}^{2}\right)\,{x}^{2}-8\,y\,{x}^{3} & -6\,{x}^{3}\,y-2\,{x}^{4}+12\,{x}^{3} \\
	\end{bmatrix}\\
	\mathbf{G}(0,0) =
	\begin{bmatrix}
		0 & 0 \\
		0 & 0 \\
	\end{bmatrix},
	\mathbf{G}(0,2) =
	\begin{bmatrix}
		0 & 0 \\
		0 & 0 \\
	\end{bmatrix},\\
	\mathbf{G}(3,0) =
	\begin{bmatrix}
		0 & 0 \\
		0 & -2\cdot 3^4+12\cdot 3^3 \\
	\end{bmatrix} =
	\begin{bmatrix}
		0 & 0 \\
		0 & -162+324=162 \\
	\end{bmatrix}
\end{align}
\begin{dmath}
	\mathbf{G}(3,2) =
	\begin{bmatrix}
		(36\cdot 2^2-6\cdot 2^3)\cdot 3-12\cdot 2^2\cdot 3^2 & (36\cdot 2-9\cdot 2^2)\cdot 3^2-8\cdot 2\cdot 3^3 \\
		(36\cdot 2-9\cdot 2^2)\cdot 3^2-8\cdot 2\cdot 3^3 & -6\cdot 3^3\cdot 2-2\cdot 3^4+12\cdot 3^3 \\
	\end{bmatrix} =
	\begin{bmatrix}
		(36\cdot 4-6\cdot 8)\cdot 3-12\cdot 4\cdot 9 & (72-9\cdot 4)\cdot 9-8\cdot 2\cdot 27 \\
		(72-9\cdot 4)\cdot 9-8\cdot 2\cdot 27 & -6\cdot 27\cdot 2-2\cdot 81+12\cdot 27 \\
		% \left(72-9\cdot 4\right)\cdot 9-8\cdot 2\cdot 81 & -6\cdot {3}^{3}\cdot 2-2\cdot {3}^{4}+12\cdot {3}^{3} \\
	\end{bmatrix}
\end{dmath}
\begin{equation}
	\Delta\mathbf{G}(3,2) =
	\begin{bmatrix}
		\overbrace{-144}^{<0, max} & -108 \\
		-108 & -162 \\
	\end{bmatrix}
	144\cdot 162-108^2 > 0
\end{equation}

\Answer{
У точці $A_4(3,2)$ --- максимум.
Точки $A_1(0,0), A_2(0,2), A_3(3,0)$ потрібно досліджувати за допомогою
похідних вищих порядків.
}

\end{document}
