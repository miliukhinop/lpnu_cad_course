\documentclass[../rgr1.tex]{subfiles}

\begin{document}


\Problem{
	Обчислити довжину дуги кривої
$x=a(t-\sin t), y=a(1-\cos t), y=0, 0\leq t\leq2\pi$
}
\Solution

Зображення кривої наведене на рис. \ref{cycloid}

\begin{equation}
	x'(t) = a(1-\cos t),~
	y'(t) = a\sin t
\end{equation}

\begin{equation}
	1-\cos(t) = 2\sin^2\frac{t}{2}
\end{equation}

\begin{dmath}
	l = \int_\alpha^\beta \sqrt{(x'(t))^2+(y'(t))^2}\,dt
	= \int_0^{2\pi} \sqrt{a^2(1-\cos t)^2 + a^2\sin^2t}
	= a\int_0^{2\pi} \sqrt{(1-\cos t)^2 + \sin^2t}
	= a\int_0^{2\pi} \sqrt{1-2\cos t +\cos^2t + \sin^2t}
	= a\int_0^{2\pi} \sqrt{2-2\cos t}
	% = a\int_0^{2\pi} 2\sqrt{\sin^2\frac{t}{2}}
	\\
	% a\sqrt{2-2\cos t}
	% https://www.wolframalpha.com/input?i=a%5Cint_0%5E%7B2%5Cpi%7D+%5Csqrt%7B2-2%5Ccos+t%7D
\end{dmath}

\paragraph{Універсальна триг. підстановка}
\begin{varwidth}{1in}
	\fbox
{
	$
	\def\arraystretch{2}\begin{array}{c|c}x=\tg\left(\frac{t}{2}\right)&\cos\left(t\right)=\dfrac{1-{x}^{2}}{{x}^{2}+1}\\\mathrm{d}x=\dfrac{1}{2\,\cos^{2}\left(\frac{t}{2}\right)}\,\mathrm{d}t&\cos^{2}\left(\frac{t}{2}\right)=\dfrac{1}{{x}^{2}+1}\end{array}
		$
	% $\def\arraystretch{2}\begin{array}{c|c}t=\tg\frac{x}{2}&\sin x=\frac{2\,t}{{t}^{2}+1}\\\mathrm{d}t=\frac{1}{2\,\cos^{2}\frac{x}{2}}\,\mathrm{d}x&\cos^{2}\frac{x}{2}=\frac{1}{{t}^{2}+1}\end{array}$
}
\end{varwidth}

\begin{dmath}
	\int{\dfrac{4\,\left|x\right|}{\left({{x}^{2}+1}\right)^{\frac{3}{2}}}}{\;\mathrm{d}x} =
	% \int{\dfrac{4\,\left|x\right|}{\left({{x}^{2}+1}\right)^{\frac{3}{2}}}}{\;\mathrm{d}(x^2+1)} \\
	\int{\dfrac{2}{\left({{x}^{2}+1}\right)^{\frac{3}{2}}}}{\;\mathrm{d}(x^2+1)}
	=2\int{\dfrac{1}{\left({{x}^{2}+1}\right)^{\frac{3}{2}}}}{\;\mathrm{d}(x^2+1)}
	=2*\frac{2}{\sqrt{x^2+1}}+C
\end{dmath}
\begin{dmath}
	a\int_0^{2\pi} \sqrt{2-2\cos t}=
	2a\int_\pi^{2\pi} \sqrt{2-2\cos t}=
	2a*2*\frac{2}{\sqrt{x^2+1}}=
	4a*\frac{2}{\sqrt{\tg^2\frac{2\pi}{2}+1}}=
	-\underbrace{4a*\frac{2}{\sqrt{\tg^2\frac{\pi}{2}+1}}}_{\to 0} = 8a
\end{dmath}


\Answer{
	8a
}

\end{document}
