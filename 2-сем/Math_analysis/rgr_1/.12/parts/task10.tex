\documentclass[../rgr1.tex]{subfiles}

\begin{document}

\Problem{Дослідити функцію на неперервність \label{continuity}
}

\paragraph{
	а) $y = \ln |\cos x|$
}

\Solution

Щоб дослідити функцію на неперервність, потрібно з'ясувати
її ОДЗ та в разі розриву з'ясувати його рід.

ОДЗ $\cos x: (-\infty; +\infty)$

ОДЗ $\ln x: (0; +\infty)$

Оскільки $\cos x$ береться за модулем, пропонована функція перетворює
всі можливі значення $x$ на множині $\mathbb R,$ окрім точок, де
$\cos x = 0.$ Множину цих точок можна записати так: $x \neq \frac{\pi}{2} \pm n\pi, n \in \mathbb Z.$

\paragraph{
	б) $y =
	\begin{cases}
		2x, & x < \pi/4 \\
		\frac{\pi}{2}\tg x, & x \geq \pi/4
	\end{cases}
	$
}

\Solution

% $$
% x \in (-\infty; \frac{\pi}{4})
% $$

ОДЗ $\tg x$: $x \in \mathbb R~|~
x \neq \pm \frac{\pi}{2} + n\pi, n \in \mathbb Z$.

Знайдімо границі
функції в точці $\frac{\pi}{2}$, аби визначити рід розриву.

\begin{equation}
	\lim_{x\to\frac{\pi}{2}+0}\frac{\pi}{2}\tg{x} =
	\lim_{x\to\frac{\pi}{2}+0}\frac{\pi}{2}\frac{\sin x}{\cos x} =
	\left\{ \frac{1}{-0} \right\} = -\infty.
\end{equation}

\begin{equation}
	\lim_{x\to\frac{\pi}{2}-0}\frac{\pi}{2}\tg{x} =
	\lim_{x\to\frac{\pi}{2}-0}\frac{\pi}{2}\frac{\sin x}{\cos x} =
	\left\{ \frac{1}{+0} \right\} = +\infty.
\end{equation}

 Оскільки границі функції в точці дорівнють $\pm\infty$, у точці $\{\frac{\pi}{2}\}$ --- розрив 2-го роду.

 % Розгляньмо точку $\{-\frac{\pi}{2}\}$:

% \begin{equation}
% 	\lim_{x\to-\frac{\pi}{2}+0}\frac{\pi}{2}\tg{x} =
% 	\lim_{x\to-\frac{\pi}{2}+0}\frac{\pi}{2}\frac{\sin x}{\cos x} =
% 	\left\{ \frac{1}{+0} \right\} = +\infty.
% \end{equation}

% \begin{equation}
% 	\lim_{x\to-\frac{\pi}{2}-0}\frac{\pi}{2}\tg{x}
% 	\lim_{x\to-\frac{\pi}{2}-0}\frac{\pi}{2}\frac{\sin x}{\cos x} =
% 	\left\{ \frac{1}{+0} \right\} = -\infty.
% \end{equation}

% У точці $\{-\frac{\pi}{2}\}$ --- теж розрив 2-го роду.

Оскільки $\tg$ --- функція з періодом $\pi$,
такі розриви присутні у всіх точках
$$
x = \pm \frac{\pi}{2} + n\pi, n \in \mathbb Z.
$$

Згідно з умовою, до значення аргументу $\frac{\pi}{4}$ функція
визначається як $2x$, тому в ній присутні тільки розриви в точках
$x = \frac{\pi}{2} + n\pi, n \in \mathbb Z.$

%$x \in (-\infty; \frac{\pi}{2})\cup{x \in e}$ <++>

\Answer{
	\paragraph{а)} $x \in \mathbb R |$ $x \neq \frac{\pi}{2} \pm n\pi, n \in \mathbb Z.$
	\paragraph{б)} Функція визначена на множині $\mathbb R$, окрім
	точок $x = \frac{\pi}{2} + n\pi, n \in \mathbb Z$, де
	присутні розриви 2-го роду.
}
\end{document}
