\documentclass[../rgr1.tex]{subfiles}

\begin{document}

\Problem{ Обчислити границю функції
$
\lim_{x\to1}\frac{x^2-2x+1}{2x^2-x-1}
$
}
\Solution

\begin{equation}
	\lim_{x\to1}\frac{x^2-2x+1}{2x^2-x-1} =
	\lim_{x\to1}\frac{(x-1)^2}{2x^2-x-1}
\end{equation}

\newcommand{\dropsign}[1]{\smash{\llap{\raisebox{-.5\normalbaselineskip}{$#1$\hspace{2\arraycolsep}}}}}%

\begin{equation}
  \begin{array}{r|r}
	\dropsign{-}
	  	2x^2 - x - 1	& x - 1 \phantom{2} \\ \cline{2-2}
		2x^2 - 2x \phantom{-1}	& 2x+1 \\ \cline{1-1} \\[\dimexpr-\normalbaselineskip+\jot]
	\dropsign{-}
		x-1\\
		x-1 \\ \cline{1-1} \\[\dimexpr-\normalbaselineskip+\jot]
		0
  \end{array} \implies
	\lim_{x\to1}\frac{x^2-2x+1}{2x^2-x-1} =
	\lim_{x\to1}\frac{x-1}{2x+1} =
	\frac{1-1}{2+1} = 0
\end{equation}


\Answer{
	0
}

\Problem{ Обчислити границю функції
	$
	\lim_{x\to -8}\frac{ \sqrt{1-x}-3 }{ 2+\sqrt[3]{x} }
	$
}
\Solution

\begin{dmath}
	\lim_{x\to -8}\frac{ \sqrt{1-x}-3 }{ 2+\sqrt[3]{x} }
	= \left\{ \frac{0}{0} \right\} =
	\lim_{x\to -8}\frac{ (1-x)-9 }{ (2+\sqrt[3]{x})(\sqrt{1-x}+3) } =
	\lim_{x\to -8}\frac{ -\cancel{(x+8)}(4-2\sqrt[3]{x} +\sqrt[3]{x^2}) }{ \cancel{(x+8)}(\sqrt{1-x}+3) } =
	\frac{2(-2) - 4 - 4}{3+3} = \frac{-12}{6} = -2
\end{dmath}

\Answer{
	$-2$
}

\Problem{ Обчислити границю функції
	$
	\lim_{x\to 1}\frac{1-x^2}{\sin{\pi x} }
	$
}
\Solution

\begin{dmath}
	\lim_{x\to 1}\frac{1-x^2}{\sin{\pi x} } =
	\left\{ \frac{0}{0} \right\}
\end{dmath}

Застосую правило Лопіталя:

\begin{dmath}
	\lim_{x\to 1}\frac{1-x^2}{\sin{\pi x} } =
	\lim_{x\to 1}\frac{-2x}{\pi\cos{\pi x} } =
	\frac{-2}{-\pi}
\end{dmath}

\Answer{
$	\frac{2}{\pi}$
}

\end{document}
