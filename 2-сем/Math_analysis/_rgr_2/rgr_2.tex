\documentclass[a4paper, 12pt, oneside]{extreport}

\input{$HOME/Templates/lpnu_doc_templates/settings/preamble}
%shell-escape
%multline, split, align*, eqnarray
\usepackage{etoolbox} % пакунок для розширеного програмування
\usepackage{amssymb} % розширені математичні символи
\usepackage{gnuplottex}

\usepackage{pgfplots}
\pgfplotsset{compat=newest,
	xlabel={$x$},
    	ylabel={$y$},
	axis lines = middle,
	axis equal
}
\usepgfplotslibrary{fillbetween}
\usetikzlibrary{patterns}

\usetikzlibrary{patterns.meta}
\usepackage{tikz-3dplot}
\tdplotsetmaincoords{60}{115} %to set up a desired angle of visualization

\usepackage{boxedminipage}
\usepackage{varwidth}
\fboxrule=1pt

%	TODO:	REFACTORING	%
\newtoggle{Report}
\togglefalse{Report}

\newtoggle{RGR}
\toggletrue{RGR}
\renewcommand\Department{ОМП}

\newcommand\Type{\RGR}
\newcommand\Number{№2}
\newcommand\Discipline{Математичний аналіз та диференціальні рівняння}

\newcommand\Class{студент групи \Group}
\newcommand\Author{\Lname~\Initials}
\newcommand\Position{доцент}
\newcommand\Instructor{Пахолок Б. Б.}
\newcommand\Variant{11}

% зміни в лічильниках підсекцій
\renewcommand{\thesubsection}{\thechapter .\arabic{subsection}}

% \usepackage{macrolist}
% \macronewlist{Answers}

% ЗАВДАННЯ
\newcommand{\Problem}{\subsection}
% РОЗВ'ЯЗОК
\newcommand{\Solution}{\subsubsection*{\centering \scshape Розв'язок:}}
% ВІДПОВІДЬ
\newcommand{\Answer}[1]{
\medskip
\null\hfill
\begin{boxedminipage}{\textwidth}
	\paragraph{Відповідь: }{#1}
	% \macrolistadd{Answers}{#1}
\end{boxedminipage}
}

\setcounter{secnumdepth}{2}

\newenvironment{compactlist}{
\begin{list}{{$\bullet$}}{
\setlength\partopsep{0pt}
\setlength\parskip{0pt}
\setlength\parsep{0pt}
\setlength\topsep{0pt}
\setlength\itemsep{0pt}
}
}{
\end{list}
}
\newenvironment{cenum}{
\begin{enumerate}{
\setlength\partopsep{0pt}
\setlength\parskip{0pt}
\setlength\parsep{0pt}
\setlength\topsep{0pt}
\setlength\itemsep{0pt}
}
}{
\end{enumerate}
}

\begin{document}

\newgeometry{top=1cm,bottom=1cm,right=1.5cm,left=1.5cm}
\newcommand{\LINE}{\rule{\linewidth}{0.4mm}}
\begin{titlepage}
	\center

	\textsc{МІНІСТЕРСТВО ОСВІТИ І НАУКИ УКРАЇНИ}\\
	\textsc{НАЦІОНАЛЬНИЙ УНІВЕРСИТЕТ "`ЛЬВІВСЬКА ПОЛІТЕХНІКА"'}\\
	% \textsc{національний університет ``львівська політехніка''}\\
	\small
	\iftoggle{RGR} {}
	{
		{\Institute}\\
	}
	{кафедра \Department}\\
	[1.5cm]
	\normalsize
	%\large

	\includegraphics[scale=0.35]{$HOME/Templates/lpnu_doc_templates/lpnu_logo.png}\\[1.5cm]

	\iftoggle{Report}
	{
		\textsc{\LARGE\bfseries звіт}\\
		\textsc{ до \Type \, \No\Number}\\
	}
	{
		\textsc{\LARGE\bfseries \Type~\Number}\\
	}
до розрахунково-графічної роботи \\
  з дисципліни: "`\Discipline"'\\
  %на тему: \\

	\iftoggle{RGR}
	{
	\LINE\\[0.2cm] % норм?
	\large
		\textsc{\bfseries Варіант \Variant}\\[0cm]
	\LINE\\[1cm]
	}
	{
	\LINE\\[0.2cm]
	\large
	\textsc{\bfseries \Topic}\\[0cm]
	\normalsize
	\LINE\\[1cm]
	}


	\begin{flushright}
		\large
		\textit{Виконав}\\
		\normalsize
		\Class \; \textsc{\Author}

		\large
		\textit{Перевірив(ла)}\\
		\normalsize
		\Position \; \textsc{\Instructor}
	\end{flushright}

	\vfill
	Львів	\the\year{}
\end{titlepage}
\Margins


\chapter{Ряди}

\subfile{series/task1.tex}
\subfile{series/task2.tex}
\subfile{series/task3.tex}
\subfile{series/task4.tex}
\subfile{series/task5.tex}
\subfile{series/task6.tex}
\setcounter{subsection}{7}
\subfile{series/task8.tex}
\chapter{Кратні інтеграли}
\setcounter{subsection}{9}

\subfile{n-fold_integrals/task10.tex}
\subfile{n-fold_integrals/task11.tex}
\subfile{n-fold_integrals/task12.tex}
\subfile{n-fold_integrals/task13.tex}
\setcounter{subsection}{14}
\subfile{n-fold_integrals/task15.tex}

\chapter{Диференціальні рівняння}

\setcounter{subsection}{2}
\subfile{difeq/task3.tex}
\subfile{difeq/task4.tex}
\subfile{difeq/task5.tex}
\setcounter{subsection}{7}
\subfile{difeq/task8.tex}
\setcounter{subsection}{19}
\subfile{difeq/task20.tex}
\subfile{difeq/task21.tex}

% \macrolistforeach{Answers}{\element}{We’re printing out \textbf{\element}. }
% \macrolistjoin{Answers}{, }

\end{document}
