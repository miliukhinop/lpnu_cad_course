\documentclass[../rgr_2.tex]{subfiles}
\usepackage{mathtools}

\begin{document}

\Problem{Знайти розв'язок задачі Коші для лінійного диференціального рівняння
$
y'-\frac{2xy}{1+x^2}=1+x^2,~y(1)=3
$
}
\Solution

\begin{align}
	y'-\frac{2xy}{1+x^2}=1+x^2 \\
	y = uv \\
	\fbox{$u'v$}+uv'-\fbox{$\displaystyle\frac{2xuv}{1+x^2}$}=1+x^2 \\
	u'v-\frac{2xuv}{1+x^2}=0 \\
	v\left(u'-\frac{2xu}{1+x^2}\right)=0 \\ % v не може бути 0 бо тоді 1+x^2=0
	\frac{du}{dx}=\frac{2xu}{1+x^2}~\Big|\vdots u \\
	\int \frac{1}{u} du=\int \frac{2x}{1+x^2}\,dx
	=\int\frac{1}{1+x^2}\,d(1+x^2) \\
	\ln|u|=\ln|1+x^2| \\
	u=1+x^2
\end{align}
тепер занулюємо обрані доданки та отримуємо:
\begin{align}
	(1+x^2)v'=1+x^2 \\
	v'=1 \implies v = x+C
\end{align}
% блін як він там записав так
\begin{align}
	y=(1+x^2)(x+C) \\ % підставимо в мяу
	y=x+C+x^3+Cx^2 \\ % підставимо в мяу
	y'-\frac{2xy}{1+x^2}=1+x^2 \\
	(x+C+x^3+Cx^2)'-\frac{2x(1+x^2)(x+C)}{1+x^2}=1+x^2 \\
	1+3x^2+2Cx-2x(x+C)=1+x^2 \\
	3x^2+2Cx-2x^2-2Cx=x^2 % усе зібралося
\end{align}

Підставмо значення та знайдімо розв'язок задачі Коші:

\begin{equation}
	3=C(1+1)+2
	\implies 3=2C+2
	\implies C=\frac{1}{2}
\end{equation}

\Answer{
$
	y=x^3+\frac{x^2}{2}+x+\frac{1}{2}
$
}

\end{document}
